% !TEX root = ../main.tex

\clearpage
\section{Terminology} \label{sec:terminology}

The terminology used in this report is specified below. The terminology differs
a bit from \\ \cite{doerr2017a}. It takes inspiration from other sources
(\cite{thede2014,tutpoint}) and follows a more biological narrative.

\begin{description}

    \item[Break point] Discussed and explained in
    section~\ref{sec:introduction}. Also refereed to as \textit{structural
    breaks}.

    \item[Allele] An allele is an entity carrying the information of break
    points. An allele stores the information about the location of a break point
    and the model information relevant for a given fitness method. (See
    explanation for fitness method a bit further down). 

    \item[Genome] A genome is an array of alleles. A \textit{gene} indicates the
    index of an allele in the genome. A genome is analogous to an array of
    integers like so:
    \begin{verbatim}
intArray[index] = integer
genome[gene] = allele
    \end{verbatim}

    \item[Individual] An individual is a possible solution to the problem at
    hand. The individual consists of a genome and possible other information
    relevant for the genetic algorithm. 

    \item[Population] A group of individuals. 

    \item[Fitness function] A function that measures how well the solution from
    an individual fits with the data. This can be any function relevant for a
    particular problem. 

    \item[Fitness method] The fitness function is dependant on the fitness
    method. The method in this project is the rectangle method mentioned in
    \cite{compstat14} and \newline \cite{doerr2017a}. 

    \item[Parent and offspring individual] Since genetic algorithms mimic
    evolution, procedures must include parent and offspring individuals. A
    parent individual is a parent to the offspring individual, meaning that any
    procedure must take one (or two) parent individual(s) and the procedure
    creates an offspring. 

    \item[Crossover (Procedure)] Crossover-procedures takes two parent
    individuals and creates an offspring from the genomes of the two parents.
    This project will incorporate \textit{one-point crossover} and
    \textit{uniform crossover}. 
    
    One-point crossover extracts the genes in the interval $0$ to a random gene
    $i - 1$ from the first parent. Then, it appends the genome with the genes
    $i$ to the last gene $n - 1$ from the second parent. Thus, the offspring's
    genome consists of the first $i$ genes from parent one and $n-i$ last genes
    from parent two. 

    Uniform crossover is more simple. For each gene in the offspring's genome,
    there is a fifty-fifty chance whether it will be the gene from parent one or
    the gene from parent two. 

    \item[Mutation (Procedure)] Just like the corona virus, the genome of an
    individual can mutate. It means that, for any gene, there is a random chance
    that the allele will change. In this case, the change will be either
    creating break points or removing them. 

\end{description}
