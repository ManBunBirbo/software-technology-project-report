% !TEX root = ../main.tex

\section{Conclusion}

Of the three goals mentioned in section~\ref{sec:goals}, all three have been
reached. The algorithm works and it is fast due to range trees. A GUI is
implemented which can show the time series with fitness markers/break points
markers. Furthermore, the user can tweak a multitude of parameters from the GUI.
The bonus goal, to make it is easy to implement further fitness methods is also
achieved through the use of abstract classes. All goals are met. 

On the non-project-goals side I have also succeeded. I have fought hard and long
with Maven, but at some point, it finally wanted to cooperate. The last goal,
about working on a big project, is perhaps the least concrete and thus difficult to
measure. But, I have learned a lot about working on big projects and made many
mistakes doing the months. And I guess that counts as making it. 

One of the key words for the application was stability. While the GUI is indeed
stable and works great, it would have been preferable to ensure stability of the
program. The problem lies within error handling: The use of it is miniscule.
This has not been a focus during the implementation fase. This is not
\textit{that} noticeable in the final product due to a stable and limiting GUI.
But error handling is one aspect to improve. 

If I could start again, I would have waited longer before coding. I rather
quickly jumped into the coding fase without having much overview of the design.
This lead to hours of aimless typing and waisted time. I should have stayed in
the design fase for longer. 

Another thing I would improve in my workflow is writing the report while coding.
I kept telling myself I would do it, but I "just had to finish this last thing"
until it was finished and I found another thing to improve. In the future,
writing the report must be higher on my daily to-do list. 

During the project, I have learned a lot about my own perfectionism and how to
limit it. I have tracked my time, so I have not spent more than 9 hours on the
project each week (or day during the last three weeks). During the last fase, I
have wanted to tweak everything \textit{one last time} and make it perfect. But
I have been good to stop those thoughts. The perfectionism is still there but it
is managed. And I am proud of the product. 