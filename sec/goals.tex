% !TEX root = ../main.tex

\section{Goals} \label{sec:goals}

A prioritized list of the functional goals for the project is listed below.
While adding a lot of features seems very appealing, I will in this project
focus on usability, stability, and speed. This leads to a somewhat conservative
list of potential features, but the quality of the final product will reflect
this decision. 

\begin{enumerate}
    \item Implement the algorithm using the rectangle-method from
    \cite{doerr2017a} in Java. 
    
    \item Visualize two-dimensional time series graphs together with the
    rectangles produced by the algorithm in a simple graphical user interface
    (GUI). This GUI will allow user to load a time series data file and see the
    output of the algorithm on the time series. The user will also be able to
    tweak certain parameters of the algorithm. 

    \item Make the algorithm more flexible by allowing the user to alter the
    values of algorithm-parameters in the GUI. 
    
    \item Optimize the runtime of the algorithm with range trees. 
\end{enumerate}

A previous goal was also to implement further fitness functions (see explanation
for \textit{fitness functions} in section~\ref{sec:terminology}). Since this
project ended up being a one-man-project, this goal was removed. In stead, a
priority is to design the project so that it is easy to implement other fitness
methods. This will be discussed further in section~\ref{sec:analysis-design}.


\subsection{Non-project goals} 

As for non-project goals, there are a few: 
\begin{itemize}

    \item Learn the Maven project structure for Java. While previous courses
    have dealt with Maven a little bit, it has never been fleshed out. It seems
    to be a structure that is widely used and thus a good system to learn. 

    \item Working on big project. This is the first big project I am working on.
    A big focus here is on the project management, report writing workflow and
    keeping track of sources in a bibliography. Especially the report writing
    workflow can become crucial, as I have a tendency to postpone it to the very
    last minute. Here, I will make it a part of my weekly work. 

\end{itemize}

My ambition level is quite high; I am a perfectionist to the core. While I will
attempt to keep the perfectionism to a minimum, I like working on bigger
projects and making it work well. This will probably result in me working a lot
on this project, purely because it will be fun, and I like improving my
less-than-optimal solutions. I am aware that this course is only 5 ECTS points
and will thus keep track of my hours spent on the project as to not overdo it. 

