% !TEX root = ../main.tex

\section{Statusrapport 1.5}

En ekstraordinær statusrapport som følge af gruppemedlem er sprunget fra. 

\subsection{Revideret projektplan}

I den første projektplan blev der lagt vægt på, at Johan havde bedre overblik
over matematikken og derfor stod for størstedelen af algoritmen. Han fik skrevet
en del på algoritmen, men han nåede aldrig at skrive den helt færdig. I stedet
for at færdiggøre hans algoritme besluttede jeg mig for at researche genetiske
algoritmer selv og implementere algoritmen helt på ny. Dette har givet mig en
rigtig god forståelse for genetiske algoritmer og bedre overblik over koden. 

\subsubsection{Skift i fokus}

Grundet jeg nu er alene (og man \textit{undtagelsesvis} måtte være to i gruppen)
må jeg begrænse arbejdet. Derfor dropper at implementere andre
fitness-funktioner for i stedet at fokusere på at optimere algoritmens
hastighed med \textit{range trees}. Mit fokus vil således ligge på følgende: 

\begin{enumerate}
    \item Implementere algoritmen der virker.  
    \item Implementere GUI, hvor parametre kan ændres og graf kan ses. 
    \item Optimere algoritmens hastighed. 
\end{enumerate}

Sideløbende med ovenstående vil jeg skrive på rapporten, hvor listen ovenfor er
så kort. Rapporten vil have høj prioritet. 

\subsubsection{Revideret tidsplan}

Hermed en revideret tidsplan. Det er indforstået, at der arbejdes på rapporten
sideløbende med alle de punkter, hvor det ikke udspecificeres. 

\begin{itemize}
    \item \textbf{Uge 9 (nu):} Lav "Statusrapport 1.5" og arbejd på at få
    algoritme til at virke. 
    \item \textbf{Uge 10:} Færdiggør implementation af algoritme som i
    pseudo-kode. 
    \item \textbf{Uge 11:} Arbejd på GUI. 
    \item \textbf{Uge 12:} Færdiggør GUI, begynd at optimere algoritme. 
    \item \textbf{Uge 13:} Fortsæt arbejde med at optimere algoritme. Dokumentér
    resultater. \textbf{Deadlines:} Send udkast af rapport til Paul (som aftalt) 
    og aflever "Statusrapport 2". 
    \item \textbf{3-ugers:} Fokus på rapport. Få arbejdet på ting fra 13-ugers
    perioden, som ikke virker tilfredsstillende endnu. 
\end{itemize}

\subsection{Statusrapport}

Algoritmen er næste færdigimplementeret. Som følge af ikke at skulle
implementere flere fitness-metoder er koden omstruktureret en smule til at gøre
det en smule mere overskuelig. Regner med at have den på plads i næste uge. 

GUI skal nok gå relativt hurtigt: Jeg har undersøgt, hvordan jeg kan sætte
firkanter på grafer, og så laver jeg selve GUI med "Scene Builder", så det
forhåbentlig ikke tager så lang tid. 

Optimering har jeg ikke kigget på endnu. Det vil kræve en del research i første
omgang og så skal der også bruges tid på implementeringen. Men det tror jeg
sagtens der er tid til med skiftet i fokus. 

Rapportskrivningen går fremad. Jeg har taget mange noter i et andet program, som
bare skal skrives rent her, hvorfor dette dokument synes tomt. Fremadrettet vil
jeg begynde at skrive ting direkte i dette dokument, så jeg ikke skal skrive
ting to gange. 

Mit største problem ligger i perfektionisme. Det har været fedt at få afklaret
forventninger med Paul, og det har givet ro på. At gøre fokus meget snævert har
også været en kæmpe hjælp. Jeg er sikker på, at jeg skal komme i mål uden at
bruge for mange timer på det. 

\newpage 