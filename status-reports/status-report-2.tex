% !TEX root = ../main.tex

\section{Statusrapport 2 (Deadline: 7. maj 2021)} \label{sec:status-rapport-2}

\subsection{Tidsplan}

Tidsplanen fra min Statusrapport 1.5 (sektion~\ref{sec:status-report-1.5})
følges nogenlunde. På dette tidspunkt skulle GUI være nogenlunde
færdigimplementeret, og jeg skulle være begyndt på at optimere algoritmen ved
brug af \textit{range trees}. Det tog en del længere tid at vise firkanter på en
graf i JavaFX end jeg havde regnet med, så jeg er endnu ikke begyndt at optimere
på algoritme. Så jeg er en smule bag tidsplanen. 

Tidsplanen er dog ikke et kæmpe problem, da jeg kun har én eksamen i maj. Jeg
har kun 15 ECTS-point dette 13-ugers, hvor 10 af dem er projektkurser. Derfor
vil jeg have tid i eksamensperioden til at kigge lidt på fagprojekt. 

\subsection{Algoritmen}

I Statusrapport 1.5 konkluderede jeg, at algoritmen spillede. Hvad jeg ikke
vidste var, at \textit{mutation}-metoden ikke blev kaldt nogensinde. Den skaber
nu nogle problemer med at generere for mange break points. For nogle
\textit{time series} er "den bedste løsning" den \textit{solution string}, hvor
næsten alle punkter er markeret som et break point. Det er ikke særlig godt, og
det kan måske bare rettes med at ændre fitness-funktionen: Implementere en anden
udgave af $p(k)$ fra \cite{doerr2017a}. Så algoritmen spiller ikke fuldstændig
lige nu. 

\subsection{Fremadrettet}

Jeg har kun brugt halvdelen af den normerede tid på kurset. Jeg har nogle gange
en følelse af, at jeg er bagud, men jeg har jo indtil videre kun brugt halvdelen
af den allokerede tid til kurset. Jeg har stadig masser af tid i 3-ugers til at
færdiggøre projektet. Det skal jeg huske på. 

Jeg glemmer at skrive på rapporten. Det er jo nogle gange sjovere at sidde og
kode, end det er at skrive på rapporten. Men rapporten \textit{er} virkelig
vigtig. Jeg skal huske at have rapporten som en prioritet. 

Jeg er lidt i tvivl om hvordan programmet skal behandle flere dimensioner. Jeg
har ikke leget rundt med det endnu. Hvis jeg har tid, kunne det være super fedt
at bruge det til noget, men det kan være, at jeg kun læser én dimension ind fra
datafiler (og gør brugeren opmærksom på det). 

\newpage